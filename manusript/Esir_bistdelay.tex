%% ****** Start of file apstemplate.tex ****** %
%%
%%
%%   This file is part of the APS files in the REVTeX 4 distribution.
%%   Version 4.1r of REVTeX, August 2010
%%
%%
%%   Copyright (c) 2001, 2009, 2010 The American Physical Society.
%%
%%   See the REVTeX 4 README file for restrictions and more information.
%%
%
% This is a template for producing manuscripts for use with REVTEX 4.0
% Copy this file to another name and then work on that file.
% That way, you always have this original template file to use.
%
% Group addresses by affiliation; use superscriptaddress for long
% author lists, or if there are many overlapping affiliations.
% For Phys. Rev. appearance, change preprint to twocolumn.
% Choose pra, prb, prc, prd, pre, prl, prstab, prstper, or rmp for journal
%  Add 'draft' option to mark overfull boxes with black boxes
%  Add 'showpacs' option to make PACS codes appear
%  Add 'showkeys' option to make keywords appear
\documentclass[aps,pre,twocolumn,groupedaddress]{revtex4-1}
%\documentclass[aps,prl,preprint,superscriptaddress]{revtex4-1}
%\documentclass[aps,prl,reprint,groupedaddress]{revtex4-1}

% amsmath package, useful for mathematical formulas
\usepackage{amsmath}
% amssymb package, useful for mathematical symbols
\usepackage{amssymb}

% You should use BibTeX and apsrev.bst for references
% Choosing a journal automatically selects the correct APS
% BibTeX style file (bst file), so only uncomment the line
% below if necessary.
%\bibliographystyle{apsrev4-1}

\begin{document}

% Use the \preprint command to place your local institutional report
% number in the upper righthand corner of the title page in preprint mode.
% Multiple \preprint commands are allowed.
% Use the 'preprintnumbers' class option to override journal defaults
% to display numbers if necessary
%\preprint{}

%Title of paper
\title{Influence of coupling delays to dynamics of reciprocally coupled bistable Hodgkin-Huxley neurons in the presence of noise}

% repeat the \author .. \affiliation  etc. as needed
% \email, \thanks, \homepage, \altaffiliation all apply to the current
% author. Explanatory text should go in the []'s, actual e-mail
% address or url should go in the {}'s for \email and \homepage.
% Please use the appropriate macro foreach each type of information

% \affiliation command applies to all authors since the last
% \affiliation command. The \affiliation command should follow the
% other information
% \affiliation can be followed by \email, \homepage, \thanks as well.
%\author{}
%\email[]{Your e-mail address}
%\homepage[]{Your web page}
%\thanks{}
%\altaffiliation{}
%\affiliation{}

\author{Pavel M. Esir}
\email[]{esir@neuro.nnov.ru}
%\homepage[]{Your web page}
%\thanks{}
\altaffiliation{Nizhny Novgorod Neuroscience Center, N.I. Lobachevsky State University of Nizhny Novgorod, Nizhny Novgorod, Russia}
\affiliation{Department of theory of oscillations and automatic control, N.I. Lobachevsky State University of Nizhny Novgorod, Nizhny Novgorod, Russia}

\author{Alexander Yu. Simonov}
\email[]{simonov@neuro.nnov.ru}
%\homepage[]{Your web page}
%\thanks{}
\altaffiliation{Nizhny Novgorod Neuroscience Center, N.I. Lobachevsky State University of Nizhny Novgorod, Nizhny Novgorod, Russia}
\affiliation{Department of theory of oscillations and automatic control, N.I. Lobachevsky State University of Nizhny Novgorod, Nizhny Novgorod, Russia}

%Collaboration name if desired (requires use of superscriptaddress
%option in \documentclass). \noaffiliation is required (may also be
%used with the \author command).
%\collaboration can be followed by \email, \homepage, \thanks as well.
%\collaboration{}
%\noaffiliation

\date{\today}

\begin{abstract}
Numerous of studies of biological neural network have found subclass of neurons which shows bistable dynamics. 
Functional role of such a neurons is not clarified by it's probably that they play important role in rhythm formation and working memory. Also connections between different neurons into the brain have delays which can play important role in formation of rhythms. In most cases when networks of neurons are simulated delays are eliminated. In this study we focused on numerical study of simple chain of 2 reciprocally coupled Hodgkin-Huxley neurons in the presence of noise. We show that at a certain value of delay regime of synchronized activity can be  more robust against the noise than where is no delays. To show this effect we use generalised two stimulus Phase Response Curve. 

\end{abstract}

% insert suggested PACS numbers in braces on next line
\pacs{}
% insert suggested keywords - APS authors don't need to do this
%\keywords{}

%\maketitle must follow title, authors, abstract, \pacs, and \keywords
\maketitle

% body of paper here - Use proper section commands
% References should be done using the \cite, \ref, and \label commands
\section{Model}
\label{s:model}

\subsection*{Neuron model}
\label{ss:neuron}

We used well-known original Hodgkin-Huxley model. The dynamics of the membrane potential, $V$, obey the following equation:

\begin{equation}
	\label{eq:HH_main}
    C \frac{dV}{dt} = -I_{Na} - I_{K} - I_{leak} + I_{e} + I_{syn};
\end{equation} 
where $C = 1 pF$ is the membrane capacitance, $t$ is time in ms, $I_{Na}, I_{K}, I_{leak}$ are sodium, potassium and leakage (chlorine) ionic currents respectively:

\begin{equation}
	\label{eq:HH_currents}
	\begin{aligned}
    I_{Na} = g_{Na}m^3h(V_m-E_{Na}); \\
    I_{K} =   g_Kn^4(V-E_K); \\
    I_{leak} = g_{leak}(V-E_{leak});  \\
	\end{aligned}
\end{equation}     
    
where non-linear voltage-dependent conductances for sodium and potassium currents are determined by states of the gating variables:    
 
 \begin{equation}
    \label{eq:gating_variables}
    \begin{aligned}
    \frac{dx}{dt} = \alpha_{x}(1-x)-\beta_{x}x, \;\;\;  x=m,n,h. \\
    \end{aligned}
\end{equation}

We take standard values for the maximal ionic conductances $g_{Na} = 120 nS$, $g_K = 36 nS$, $g_{leak} = 0.3 nS$, the reversal potentials $E_{Na} = 55 mV$, $E_K = - 77 mV$, $E_{leak} = -54.4 mV$. 

$I_{e}$ in (\ref{eq:HH_main}) is a constant externally applied current determining the depolarization level of the membrane and dynamical regime that can be either excitable, oscillatory or bistable. $I_{syn}$ is the synaptic current implementing interactions between neurons describing in the next subsection.

\subsection*{Synaptic interaction}
\label{ss:synapse}
Synaptic currents are described by the following equations:

\begin{equation}
	\label{eq:I_syn}
	\begin{aligned}
    &\frac{dc}{dt} = - \frac{c}{\tau_{syn}} + \frac{w e}{\tau_{syn}} \delta(t-t_{sp} + d);\\
    &\frac{dI_{syn}}{dt} = c - \frac{I_{syn}}{\tau_{syn}};\\
    \end{aligned}
\end{equation}

where $c$ is the synaptic variable. The synaptic time constant $\tau_{syn} = 0.2 ms$, $w$ is the synaptic weight in pA and $e$ is the Euler's number, so the peak value of EPSC is equal $w$, $\delta$ is Dirac delta function, $t_{sp}$ is the presynaptic spike timing, $d$ is the axonal conduction delay. When a spikes arrives to a postsynaptic neuron $I_{syn}$ on it has the form of alpha function (\ref{eq:alpha}) which has a maximum $w$  at time $\tau_{syn}$

\begin{equation}
	\label{eq:alpha}
\alpha = \frac{e w t}{\tau_{syn}} e^{-\frac{t}{\tau_{syn}}}.
\end{equation} 

\section{Results}
\label{ss:results}

\subsection{Activity of 2 neurons in the presence of Poison noise}
\label{ss:switchact}


% If in two-column mode, this environment will change to single-column
% format so that long equations can be displayed. Use
% sparingly.
%\begin{widetext}
% put long equation here
%\end{widetext}

% figures should be put into the text as floats.
% Use the graphics or graphicx packages (distributed with LaTeX2e)
% and the \includegraphics macro defined in those packages.
% See the LaTeX Graphics Companion by Michel Goosens, Sebastian Rahtz,
% and Frank Mittelbach for instance.
%
% Here is an example of the general form of a figure:
% Fill in the caption in the braces of the \caption{} command. Put the label
% that you will use with \ref{} command in the braces of the \label{} command.
% Use the figure* environment if the figure should span across the
% entire page. There is no need to do explicit centering.

% \begin{figure}
% \includegraphics{}%
% \caption{\label{}}
% \end{figure}

% Surround figure environment with turnpage environment for landscape
% figure
% \begin{turnpage}
% \begin{figure}
% \includegraphics{}%
% \caption{\label{}}
% \end{figure}
% \end{turnpage}

% tables should appear as floats within the text
%
% Here is an example of the general form of a table:
% Fill in the caption in the braces of the \caption{} command. Put the label
% that you will use with \ref{} command in the braces of the \label{} command.
% Insert the column specifiers (l, r, c, d, etc.) in the empty braces of the
% \begin{tabular}{} command.
% The ruledtabular enviroment adds doubled rules to table and sets a
% reasonable default table settings.
% Use the table* environment to get a full-width table in two-column
% Add \usepackage{longtable} and the longtable (or longtable*}
% environment for nicely formatted long tables. Or use the the [H]
% placement option to break a long table (with less control than 
% in longtable).
% \begin{table}%[H] add [H] placement to break table across pages
% \caption{\label{}}
% \begin{ruledtabular}
% \begin{tabular}{}
% Lines of table here ending with \\
% \end{tabular}
% \end{ruledtabular}
% \end{table}

% Surround table environment with turnpage environment for landscape
% table
% \begin{turnpage}
% \begin{table}
% \caption{\label{}}
% \begin{ruledtabular}
% \begin{tabular}{}
% \end{tabular}
% \end{ruledtabular}
% \end{table}
% \end{turnpage}

% Specify following sections are appendices. Use \appendix* if there
% only one appendix.
%\appendix
%\section{}

% If you have acknowledgments, this puts in the proper section head.
%\begin{acknowledgments}
% put your acknowledgments here.
%\end{acknowledgments}

% Create the reference section using BibTeX:
\bibliography{basename of .bib file}

\end{document}
%
% ****** End of file apstemplate.tex ******

